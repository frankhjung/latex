% \documentclass[10pt,a4paper]{article}

% \begin{document}

\subsection*{Restore older version of a file}
\addcontentsline{toc}{subsection}{Restore older version of a file}

\noindent Given a list of commits on a file, revert to an older state of the
file and commit as a new change.

\subsubsection*{Method}
\addcontentsline{toc}{subsubsection}{Method}

\begin{itemize}
\item list commit history of file to get commit hashes
\item checkout older commit for that file
\item  check local file has changed, so can commit as new change
\end{itemize}

\subsubsection*{Detail}
\addcontentsline{toc}{subsubsection}{Detail}

\noindent This is the current version of our test file:

\begin{lstlisting}[language=ksh,caption={Test file: README.md}]
test
changed
exit-code
\end{lstlisting}

\noindent To determine where to restore to, list the commit history of file to get commit
hashes:

\begin{lstlisting}[language=ksh]
git log --oneline README.md
dc9de3e latest change
dbc5cca another change
fe66fc4 new change
c56ac1c initial
\end{lstlisting}

\noindent Lets restore to a version 2 commits back. To checkout older commit
for that file:

\begin{lstlisting}[language=ksh]
git checkout dbc5cca README.md
\end{lstlisting}

\noindent Check that the local file has changed, so can commit as new change:

\begin{lstlisting}[language=ksh,caption={Changed file: README.md}]
test
changed
exit-code
another change
\end{lstlisting}

\noindent You can now add changed file and commit this version as a new commit:

\begin{lstlisting}[language=ksh]
git status
On branch master
Changes to be committed:
  (use "git reset HEAD <file>..." to unstage)

	modified:   README.md
\end{lstlisting}

% \end{document}
